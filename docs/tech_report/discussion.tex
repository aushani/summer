\section{Discussion and Future Work} \label{sec:discussion}

While the generated detection maps show some promising results, they do exhibit
many false positives. The models considered here are not able to differentiate
between certain background structure (such as vertical walls) and objects such
as cars.

In the future, it will be interesting to more closely compare these approaches
with that of \citep{wang2015RSS} or \citep{Engelcke2017ICRA}. Both of these
methods are similar to ours in that they used gridded \ac{LIDAR} observations
and shape features (although we additionally use free space). Perhaps
investigating these methods more closely will shed light on how to eliminate the
false positive detections on background structure. Perhaps this is due to their
training scheme, in which they perform hard negative mining to actively harvest
difficult background examples.

Furthermore, there are different ways of evaluating the observation model. While
we present a probabilistic method here, the literature provides many other
techniques. In object recognition, deep learning is commonly used to classify an
occupancy-grid-like representation of some object. While a large network might
prove intractable to rely on for evaluate \eqref{eq:detection_map} exhaustively,
investigating these methods more closely might provide value insight.
Alternatively, perhaps we could find some structure to exploit in the evaluation
of these models to improve runtime performance when they are repeatedly
evaluated, much like the sparse voting scheme in \citet{wang2015RSS} or the
implementation of separable kernels.
