\section{Detection Likelihood Map}\label{sec:dlm}

Our goal is to compute a detection map that captures how likely it is for an
object to exist at a particular location in the world. As our work is targeted
for applications with autonomous vehicles, for runtime performance, we make the
assumption that the world is 2.5D. Namely, we assume that all objects exist on a
horizontal plane and their pose can be fully described by their 2D location and
rotation. However, we fully model the 3D nature of the sensor observations.

We receive a set of \ac{LIDAR} observations, $\mathbf{Z} = \{z_{1:n_z}\}$. Given
these, we evaluate:
%
\begin{align}
  p(\mathrm{obj_{c, x}}| \mathbf{Z}) \text{,} \label{eq:detection_map}
\end{align}
%
where $\mathrm{obj_{c, x}}$ denotes the presence of an object of class $c$ with
pose $x$. As we focus on KITTI, the classes we are concerned with are Car,
Pedestrian, and Cyclist. Additionally, we model a Background class as well. We
refer to \eqref{eq:detection_map} as a probabilistic detection map. Our goal is
to evaluate and generate this map.

Applying Bayes' rule, we have:
%
\begin{align}
  p(\mathrm{obj_{c, x}} | \mathbf{v}) &=
    \frac
      {p(\mathrm{obj_{c, x}}) p(\mathbf{Z} | \mathrm{obj_{c, x}})}
      {p(\mathbf{Z})}
  \text{,}
  \label{eq:bayes}
\end{align}
%
where $p(\mathrm{obj_{c, x}})$ is an object prior,
$p(\mathbf{Z} | \mathrm{obj_{c, x}})$ is the observation model, and
$p(\mathbf{Z})$ is a normalization factor.

The key part of \eqref{eq:bayes} is the observation model. We require an
observation model that is expressive enough to detect objects and differentiate
between different classes. At the same time, we must be able to evaluate this
model quickly for different classes and poses so that it is tractable to
evaluate \eqref{eq:bayes} for a large number of object poses and classes to
fully build the detection map.
